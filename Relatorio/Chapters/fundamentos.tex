\chapter{Conceito do projeto do portfólio}
\label{chap:fundteor}
%--------- NEW SECTION ----------------------

% isso é igual <=  === <> #{ #( www
% <| |>
% ===

Lista dos documentos
\begin{enumerate}
   \item diagrama de classe
   \item diagrama de casos de uso
   \item diagrama de sequência
\end{enumerate}


Neste capítulo serão abordados os requisitos do cliente, os requisistos técnicos e a pesquisa por similares. 



%conferir se precisa de requisitos do cliente
\section{Requisitos do cliente}
 O cliente definiu certos requisitos quanto à operação e  às características do portfólio:
 \begin{itemize}
    \item Portfólio simples;
    \item Informações de contato;
    \item Mensagem automática para o e-mail da cliente;
    \item Mostrar as tecnologias e linguagens que domina;
    \item Detalhes em cor de Rosa;
 \end{itemize}

 \section{Requisitos funcionais}
 
 \begin{itemize}
 \item RF001 - Exibir informações pessoaisApresentar nome, formação acadêmica, contatos (email, telefone,etc.) e umabreve descrição sobre o estudante;
 \item RF002 - Listar projetos desenvolvidosListar projetos com informações como título, descrição,tecnologias usadas elinks para repositórios (como GitHub) ou demos;
 \item RF003 - Exibir habilidades técnicasListar linguagens de programação, frameworks, bancos de dados eoutrasferramentas que o estudante domina;
 \item RF004 - Permitir contatoFormulário de contato (ou redirecionar paraemail/mensagem) para que visitantespossam enviar mensagens;
 \item RF005 - Responsividade;
 \item RF006 - Integração com redes sociais Botões/links para GitHub, LinkedIn, Instagram;
\end{itemize}



%  \section{Missão}
%  \lipsum
%  %desenvolver mais
%  Além disso, o Walker deve realizar um desafio, que consiste em navegar de forma autônoma, se localizar por meio de tags e encontrar um determinado objeto.



%  \section{Pesquisa por similares}


% %----------------------------------------------------------

% %--------- NEW SECTION ----------------------


% %---------------picture------------------------------------
% % \begin{figure}
% %     \centering
% %     \subfigure[Figure A]{\label{fig:a}\includegraphics[width=60mm]{./lq}}
% %     \subfigure[Figure B]{\label{fig:b}\includegraphics[width=60mm]{./lq}}
% %     \subfigure[Figure C]{\label{fig:c}\includegraphics[width=\textwidth]{./lq}}
% %     \caption{Three simple graphs}
% %     \label{fig:three graphs}
% % \end{figure}
% %----------------------------------------------------------

% % \begin{figure}
% %     \centering
% %     \begin{subfigure}[b]{0.3\textwidth}
% %         \centering
% %         \includegraphics[width=\textwidth]{./lq}
% %         \caption{$y=x$}
% %         \label{fig:y equals x}
% %     \end{subfigure}
% %     \hfill
% %     \begin{subfigure}[b]{0.3\textwidth}
% %         \centering
% %         \includegraphics[width=\textwidth]{./lq}
% %         \caption{$y=3sinx$}
% %         \label{fig:three sin x}
% %     \end{subfigure}
% %     \hfill
% %     \begin{subfigure}[b]{0.3\textwidth}
% %         \centering
% %         \includegraphics[width=\textwidth]{./lq}
% %         \caption{$y=5/x$}
% %         \label{fig:five over x}
% %     \end{subfigure}
% %        \caption{Three simple graphs}
% %        \label{fig:three graphs}
% % \end{figure}


% % %--------- NEW SECTION ----------------------
% % \section{Assunto 2}
% % \label{sec:ass2}
% % flkjasdlkfjasdlkfjs

% % \begin{table}[h]
% %     \begin{subtable}[h]{0.45\textwidth}
% %         \centering
% %         \begin{tabular}{l | l | l}
% %         Day & Max Temp & Min Temp \\
% %         \hline \hline
% %         Mon & 20 & 13\\
% %         Tue & 22 & 14\\
% %         Wed & 23 & 12\\
% %         Thurs & 25 & 13\\
% %         Fri & 18 & 7\\
% %         Sat & 15 & 13\\
% %         Sun & 20 & 13
% %        \end{tabular}
% %        \caption{First Week}
% %        \label{tab:week1}
% %     \end{subtable}
% %     \hfill
% %     \begin{subtable}[h]{0.45\textwidth}
% %         \centering
% %         \begin{tabular}{l | l | l}
% %         Day & Max Temp & Min Temp \\
% %         \hline \hline
% %         Mon & 17 & 11\\
% %         Tue & 16 & 10\\
% %         Wed & 14 & 8\\
% %         Thurs & 12 & 5\\
% %         Fri & 15 & 7\\
% %         Sat & 16 & 12\\
% %         Sun & 15 & 9
% %         \end{tabular}
% %         \caption{Second Week}
% %         \label{tab:week2}
% %      \end{subtable}
% %      \caption{Max and min temps recorded in the first two weeks of July}
% %      \label{tab:temps}
% % \end{table}