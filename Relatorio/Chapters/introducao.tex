\chapter{Introdução}
\label{chap:intro}

% Este pode ser um parágrafo citado por alguém \cite{Barabasi2003-1} e \cite{barabasi2003linked}.
% Para ajustar veja o comentário do capítulo \ref{chap:fundteor}.

% As orientações do robô \cite{aperea-1}.

% fakdfjlsdjfldsjfldsj
% dfkhfdskfhkdjh


% Segundo \citeonline{barabasi2003linked}, ...

% 
% \loremipsum dolor sit amet, consectetur adipiscing elit. Sed do eiusmod tempor incididunt ut labore et dolore magna aliqua. Ut enim ad minim veniam, quis nostrud exercitation ullamco laboris nisi ut aliquip ex ea commodo consequat. Duis aute irure dolor in reprehenderit in voluptate velit esse cillum dolore eu fugiat nulla pariatur. Excepteur sint occaecat cupidatat non proident, sunt in culpa qui officia deserunt mollit anim id est laborum.
%--------- NEW SECTION ----------------------
\section{Objetivos}

\label{sec:obj}

\label{sec:obj}

\subsection{Objetivos Específicos}
\label{ssec:objesp}

O projeto tem como principal objetivo o desenvolvimento de um portfólio pessoal profissional para o cliente, com foco em:
\begin{itemize}
      \item Apresentar informações pessoais e profissionais de forma clara e atrativa;
      \item Reunir informações relevantes, como experiências, habilidades técnicas, formação e projetos desenvolvidos;
      \item Facilitar o contato com recrutadores e empresas, por meio de um formulário funcional e links para redes sociais;
      \item Aumentar a presença online do cliente, com um site responsivo acessível em diferentes dispositivos;
      \item Proporcionar autonomia, permitindo futuras atualizações de conteúdo de maneira simples;
      \item Utilizar boas práticas de desenvolvimento, garantindo desempenho, acessibilidade e escalabilidade;
\end{itemize}

\subsubsection*{Objetivos específicos principais}
\label{sssec:obj-principais}



%--------- NEW SECTION ----------------------
\section{Justificativa}
\label{sec:justi}

O desenvolvimento de um portfólio pessoal se justifica por diversos fatores estratégicos e profissionais. Abaixo, estão os principais motivos que fundamentam este projeto 
\begin{itemize}
\item Fortalecimento da presença digital
\item Centralização de informações
\item Valorização da imagem profissional
\item Autonomia e flexibilidade
\item Aprendizado e desenvolvimento técnico
\end{itemize}
%--------- NEW SECTION ----------------------
\section{Organização do documento}
\label{section:organizacao}

Este documento apresenta $5$ capítulos e está estruturado da seguinte forma:

\begin{itemize}

  \item \textbf{Capítulo \ref{chap:intro} - Introdução}: Contextualiza o âmbito, no qual a pesquisa proposta está inserida. Apresenta, portanto, a definição do problema, objetivos e justificativas da pesquisa e como este \thetypeworkthree está estruturado;
  \item \textbf{Capítulo \ref{chap:fundteor} - Fundamentação Teórica}: * Fundamentação Teórica: Apresenta os principais conceitos relacio-nados ao desenvolvimento web, design responsivo, usabilidade, metodologias ágeis eferramentas utilizadas no projeto;
  \item \textbf{Capítulo \ref{chap:metod} - Materiais e Métodos}: * Materiais e Métodos: Descreve a metodologia adotada (modeloWaterfall), os processos de levantamento de requisitos, prototipação, implementação,além das ferramentas utilizadas durante o desenvolvimento;
  \item \textbf{Capítulo \ref{chap:result} - Resultados}: * Resultados: Apresenta o produto final desenvolvido, as funcionali-dades implementadas, os testes realizados, o feedback do cliente e a validação doprojeto;
  \item \textbf{Capítulo \ref{chap:conc} - Conclusão}: Apresenta as conclusões, contribuições e algumas sugestões de atividades de pesquisa a serem desenvolvidas no futuro.

\end{itemize}
